\documentclass{article}

\usepackage{xcolor}
\usepackage{amsmath}
\usepackage[a4paper]{geometry}
\usepackage[parfill]{parskip}
\usepackage[noabbrev,nameinlink,capitalise]{cleveref}
\usepackage{siunitx}
\usepackage{microtype}

\title{{Snowmelt Cup II}\\{\Huge Rules}}
\author{}
\date{}

\newcommand{\alert}[1]{{\color{red} #1}}
\newcommand{\penalyformula}[1]{{\color{cyan} #1}}
\newcommand{\startsectionfromzero}{\setcounter{section}{-1}}
\newcommand{\srbgcolor}{red!40}
\newcommand{\specialrule}[1]{\colorbox{\srbgcolor}{\parbox{\textwidth}{#1}}}

\startsectionfromzero
\frenchspacing
\renewcommand\arraystretch{1.3}

\begin{document}
\maketitle
\alert{
	Any personnels
	closely related to the tournament,
	including but not limited to
	the participants and the judges,
	are strictly prohibited from
	proceeding with any form of
	violations against the fairness of the event,
	and any form of activities that
	may cause harm to the other's legitimate interests.
	Violators may face penalties
	and take personal responsibility for the consequences.
}

\section{Application requirements} \label{sec:application-requirements}

Contestants are required to:

\begin{enumerate}
	\item Be able to record a clear handcam.
	\item Possess more than 80\% of the charts
	      in their own division.
	      This includes all charts with
	      chart constant $\geq 10$,
	      except for those of Beyond difficulty
	      AND being included in a song pack other than
	      Main Story chapters.
	      If this has become an issue,
	      please inform FLAME, the host of event in advance.
\end{enumerate}

\section{Number of contestants: 48}

\section{Formal rules}

\subsection{Divisions}

Contestants will be split into 3 divisions during the event:
Speed~(S), Technical~(T) and Complicated~Beats~(Cb).
Songs will be classified
into 3 categories correspondingly,
namely class songs.

% An explanation of 3 song categories:
\begin{table}[!htbp]
	\centering
	\begin{tabular}{cl}
		\hline
		Category & What to expect                                                                \\ \hline
		S        & Charts testing the ability to cope with high-speed note patterns              \\
		T        & Charts testing positioning and the ability to cope with complex note patterns \\
		Cb       & Charts testing the ability to cope with complex rhythm                        \\ \hline
	\end{tabular}
\end{table}

\alert{
	The specific allocation of the songs  % <!-- or 'charts'? -->
	will be revealed by the time of application.
}

Contestants will only compete with each other
within their own division,
using and ban/picking songs of their own category.
Part of the songs may be classified into multiple categories.

\subsection  {Audition}

Contestants may freely choose their desired division
during application,
by which 3 class songs for each category will be published.
Once decided, the contestants will need to play through the chosen class songs in a stretch,
and submit the handcam and the screenshots of all 3 play results.
Retrying, re-recording and editting the handcam are forbidden.  % <!-- should we emphasise that pausing is allowed ? -->
One may also switch to another division before the deadline.
If so, the contestant should inform one of the staff about the change,
and go through the same procedure once more as above,
using the new choice of songs.  % <!-- verbose -->

At the end of the audition,
contestants will be ranked
according to their scores on the class songs.
\textbf{In case of division A having $\mathbf{> 16}$ contestants
	AND B with $\mathbf{< 16}$,
	the prospective contestants eliminated from A
	will be given the opportunity to switch to B,
	but with a ranking certainly lower than
	anyone else originally placed there.}
Contestants who decide to switch
will need to go through the same
audition procedure as aforementioned,
and those ranked lower than 16 still will be eliminated
once for all.  % <!-- bad word choice -->
(Note: contestants who switch divisions
still need to possess the required songs
as stated in \cref{sec:application-requirements}.)

Otherwise, in the case of
all the divisions having $\geq 16$ contestants,
those ranked lower than 16 will be eliminated.
Contestants who passed will then be examined for
possessing the required songs.
Contestants also need to bind with Bot
using their Arcaea friend code.
Anyone not possessing the required songs
will be disqualified and being
replaced by a previously eliminated
contestant in the division,
in the order of ranking.

\subsection {Round I: Group Stage, 16 into 8}

Within the same division,
contestants will be arranged into \emph{groups} of 4.
Each group will be allocated with a judge,
who will randomly select \emph{6 non-repeating songs}
from the pool in the relevant category
for the upcoming matches.
Before the matches begins,
both competing contestants
may DM the judge to pick one and only one song,
to increase its chance of being selected
in their participating round by \qty{25}{\percent},
AND/OR ban one and only one song.
When placed on the same song,
a pick and a ban cancel out each other.
The song picked/banned must be within
the pool of the corresponding division.
Contestants may not make changes to their choices later.

The final ranking of each member in the group will be determined  % <!-- how? -->
when all the matches in the group are finished.
The top 2 contestants will advance to the next round.

Grouping of contestants is decided
by the rankings from the audition
(e.g. contestants ranked 1, 5, 9, 13
are grouped together).

\alert{
	The songs with chart constant 9.3--10.9
	are in the range of the selection for this round.
}

\subsection  {Round II: Individual matches, 8 into 4}

Every champion  % not yet champion, i guess?
from the last round
will be paired with a runner-up
from a different group for a match.

Process: Each contestant bans 3 songs →
the judge announces the banned songs →
each contestant picks 2 songs + the judge picks 1 song by random = 5 songs in total.
The contestant with the higher points advances to the next round.
(Ban/Pick must be done half an hour before the match)

\specialrule{
	Special rule: After the pick is done,
	a contestant may DM the judge with "high" or "low"
	to have a sneak about
	the approximate chart constant of the opponent's pick
	($\pm0.1$).
}

\nopagebreak

\alert{
	The songs with chart constant 9.7--11.5
	are in the range of the selection for this round.
}

\subsection{Round III: Comeback % <!-- expect better substitutes --> 
	matches \& Finals}

\paragraph{Comeback matches.} 4 eliminated contestants
from Round II will rematch to yield the top 2
(ranked 5th and 6th).

Process and \colorbox{\srbgcolor}{special rule} remains the same as Round I,
but without picking songs,
and the number of songs reduces to 3.
The range of the songs stays the same as in Round II.

\paragraph{Finals.} 4 winning contestants from Round II
are ranked from 1st to 4th place
according to their points.

Contestants placed 5th and 6th, the offensive,
will need to each challenge any
one of the top 4 contestants, the defensive.
The 5th place contestant will be prioritised to make the decision.

\paragraph{Part 1}

\begin{enumerate}
	\item  If either the 1st or 2nd place contestant is challenged:
	      if the challenger wins,
	      then two contestants swap places.
	      Else if the challenged wins, the challenger is eliminated.

	\item  If either the the 3rd or 4th place contestant is challenged:
	      if the challenger wins,
	      then two contestants swap places,
	      and the challenged is eliminated.
	      Else if the challenged wins, the challenger is eliminated.
\end{enumerate}

\paragraph{Part 2}

\begin{enumerate}

	\item If there are still 6 contestants left after Part 1,
	      then the 5th and 6th place contestants must each challenge
	      one of the 3rd or 4th place contestants,
	      following the same rules as Part 1.

	\item After the challenge, the 3rd and 4th place contestants
	      must challenge one of the 1st or 2nd place contestants.
	      The winning contestants will enter the champion match,
	      and the losing contestants will contend for the
	      second and third runner-up.
	      The procedure and special rules of which
	      are the same as Round II,
	      except two more random songs are added
	      to the pool in the champion match.

\end{enumerate}

\alert{
	The songs with chart constant 10.5--12.0
	are in the range of the selection for this round.
}

\section{Scoring}  % here I come

Note: ``score'' is your in-game play result,
while ``points'' is the score you get in the tournament,
which is derived from your play result.
``Scoring'' is the procedure of calculating the points.

\newcommand{\Points}{\text{Points}}

\subsection{Single song scoring}
% \nopagebreak
The points from one single music play is calculated as follows:
\alert{
	\begin{align*}
		% \Points		& = \frac{M\times1 + P\times0.9 + F\times0.2}{N} \\
		\Points        & = \frac{M + 0.9P + 0.2F}{N}   \\
		\text{where}~M & = \text{No.~of Max Pure,}     \\
		P              & = \text{No.~of Pure,}         \\
		F              & = \text{No.~of Far,}          \\
		N              & = \text{Total count of notes}
	\end{align*}
}

\subsection{Bonus}

Bonus conditions are division-dependent (see below).
Once triggered, an alternative scoring formula
will be used in place.  % ...as to include the bonus.

Division S: require $(S_\text{me} - S_\text{opponent} \geq \num[group-separator={,}]{10000})$, where $S = \text{Score}$
\alert{
	\begin{equation*}
		\Points = \frac{M + 0.96P + 0.2F}N
	\end{equation*}
}

Division T: require $(L_\text{me} \leq L_\text{opponent})$, where $L = \text{No.~of Lost}$
\alert{
	\begin{equation*}
		\Points = \frac{M + 0.9P + 0.5F}N
	\end{equation*}
}

Division Cb: require $(P_\text{me} - P_\text{opponent} \geq 10$ \emph{AND} $S_\text{me} \geq \num[group-separator={,}]{9970000})$
\alert{
	\begin{equation*}
		\Points		= \frac{M + 0.9P + 0.2F + 0.75L} N
	\end{equation*}
}

\subsection{Overtime penalty}
\penalyformula{
	\begin{align*}
		\text{Penalty}(song) & = \Points(song) \times \left(1-\frac{{T}^2} {3600}\right) \\
		\text{where}~T       & =\text{Overtime in seconds}
	\end{align*}
}

\subsection{Audition scoring}

Audition points summarise the single song points
for all three class songs.

\section{Time allocation}

\begin{itemize}

	\item Upon confirming the grouping,
	      contestants should reach out to their opponents
	      in order to determine the schedule of matches
	      and duly notify the judge.

	\item Should a contestant wish to Ban/Pick,
	      they need to notify the judge at least 30 minutes
	      prior to the starting of the match.

	\item During the group stage (Round I),
	      every match will commence 1 minute
	      after announcing the song.
	      5 minutes will be given for the music play,
	      during which the contestant may restart the song if wished to.
	      Contestants will then be given 4 minutes for
	      a moment of respite.
	      Although matches may be separatedly scheduled,  % how about "asynchronously"?
	      contestants within the same group are recommended
	      to finish up soon whenever possible,
	      for the convenience of statistics.  % <!-- feels weird man -->

	\item Time allocation during the individual matches (Round II) will be identical as in Round I.
\end{itemize}

\section{Handcams \& screenshots}

Handcam is required throughout the tournament.
The recording must remain unedited,
and must include timestamps
at both the beginning and end of the music play,
accurate to the second,
based on Coordinated Universal Time (UTC).

Contestants should remember to capture a timestamped screenshot
of the play result
when finishing a music play.
Therefore in case of a mistakenly ruled overtime,
the screenshot/handcam could be
submitted as evidence for the judges to review.

Only the winning contestants need to submit the handcam
at the end of the match.

Exception: one would \textbf{never}
trigger the overtime penalty
by retrying a song
for \textbf{no more than once},
regardless of whether the timeout occurs.
In case of a wrongful judgement,
please appeal to the staff team.

\section{Ban/Pick}

\begin{itemize}
	\item Ban: Disable a song from appearing in the current round.

	\item Pick: Choose a song to make it definitely
	      appearing in the current round.  % <!-- "unless otherwise stated" -->

	\item During the group stage (Round I),
	      banned and picked song will be revealed
	      at the end of the round.
	      \begin{itemize}
		      \item \alert{
			            Note: repeated picks on one
			            song will be considered as one pick only.
		            }
	      \end{itemize}

	\item During the individual matches (Round II),
	      the banned songs will be revealed before the commencement of each match,
	      from which contestants will not be able to pick.
	      Picked songs will be revealed at the end of the round.
	      \begin{itemize}
		      \item \alert{
			            Note: repeated picks on one
			            song will be considered as one pick only,
			            and a random song,
			            guaranteed not to be
			            overlapping with the pick songs,
			            will kick in.  % <!-- informal -->
		            }
	      \end{itemize}
\end{itemize}

\section{Schedule}

All the timestamps below are in form of
\texttt{yyyy/mm/dd HH:MM}, GMT+8.

\begin{itemize}
	\item   Audition: 2023/07/16 12:00--2023/07/30 00:00
	\item   Round I: 2023/08/01--2023/08/10
	\item   Round II: 2023/08/11--2023/08/18
	\item   Round III: 2023/08/19--2023/08/26
\end{itemize}

\section{Time slots}

All the timestamps below use the form of
\texttt{HH:MM}, GMT+8.

Round I/II: 8:00--9:00, 14:00--15:00, 15:30--16:30, 19:00--20:00, 20:30--21:30

Round III takes the same time slots as Round II
but with 20 more minutes appended.

Should you have any other time arrangements,
please contact the judge who will
try to accommodate as much as possible.

Please bear in mind that
the time slots do not include the Ban/Pick stage,
which will be instead scheduled 30 minutes prior to the commencement.

\section{Prizes}

\begin{table}[!htbp]
	\centering
	\begin{tabular}{ll}
		\hline
		Achievement  % bad wording!
		                 & Prize                                    \\ \hline
		Champion         & Arcaea Sound Collection Album per person \\
		First runner-up  & 1000 Memories per person                 \\
		Second runner-up & 800 Memories per person                  \\
		4th--6th place   & 500 Memories per person                  \\
		7th--8th place   & 300 Memories per person                  \\
		9th--16th place  & 200 Memories to two by random            \\ \hline
	\end{tabular}
\end{table}

Each division would yield a set of the above standings
i.e.~three champions will be produced in total,
one in each division, and so does the rest.
% ...per division

All the prizes above will be awarded in form of  % maybe 'be converted'?
cash---every Arcaea Sound Collection Album converts to ¥150
and every 100 Memories converts  % convert or converts???
to ¥6.

\end{document}
